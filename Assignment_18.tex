\documentclass[journal,12pt,twocolumn]{IEEEtran}

\usepackage{setspace}
\usepackage{gensymb}

\singlespacing


\usepackage[cmex10]{amsmath}

\usepackage{amsthm}

\usepackage{mathrsfs}
\usepackage{txfonts}
\usepackage{stfloats}
\usepackage{bm}
\usepackage{cite}
\usepackage{cases}
\usepackage{subfig}

\usepackage{longtable}
\usepackage{multirow}

\usepackage{enumitem}
\usepackage{mathtools}
\usepackage{steinmetz}
\usepackage{tikz}
\usepackage{circuitikz}
\usepackage{verbatim}
\usepackage{tfrupee}
\usepackage[breaklinks=true]{hyperref}
\usepackage{graphicx}
\usepackage{tkz-euclide}

\usetikzlibrary{calc,math}
\usepackage{listings}
    \usepackage{color}                                            %%
    \usepackage{array}                                            %%
    \usepackage{longtable}                                        %%
    \usepackage{calc}                                             %%
    \usepackage{multirow}                                         %%
    \usepackage{hhline}                                           %%
    \usepackage{ifthen}                                           %%
    \usepackage{lscape}     
\usepackage{multicol}
\usepackage{chngcntr}

\DeclareMathOperator*{\Res}{Res}

\renewcommand\thesection{\arabic{section}}
\renewcommand\thesubsection{\thesection.\arabic{subsection}}
\renewcommand\thesubsubsection{\thesubsection.\arabic{subsubsection}}

\renewcommand\thesectiondis{\arabic{section}}
\renewcommand\thesubsectiondis{\thesectiondis.\arabic{subsection}}
\renewcommand\thesubsubsectiondis{\thesubsectiondis.\arabic{subsubsection}}


\hyphenation{op-tical net-works semi-conduc-tor}
\def\inputGnumericTable{}                                 %%

\lstset{
%language=C,
frame=single, 
breaklines=true,
columns=fullflexible
}
\begin{document}


\newtheorem{theorem}{Theorem}[section]
\newtheorem{problem}{Problem}
\newtheorem{proposition}{Proposition}[section]
\newtheorem{lemma}{Lemma}[section]
\newtheorem{corollary}[theorem]{Corollary}
\newtheorem{example}{Example}[section]
\newtheorem{definition}[problem]{Definition}

\newcommand{\BEQA}{\begin{eqnarray}}
\newcommand{\EEQA}{\end{eqnarray}}
\newcommand{\define}{\stackrel{\triangle}{=}}
\bibliographystyle{IEEEtran}
\providecommand{\mbf}{\mathbf}
\providecommand{\pr}[1]{\ensuremath{\Pr\left(#1\right)}}
\providecommand{\qfunc}[1]{\ensuremath{Q\left(#1\right)}}
\providecommand{\sbrak}[1]{\ensuremath{{}\left[#1\right]}}
\providecommand{\lsbrak}[1]{\ensuremath{{}\left[#1\right.}}
\providecommand{\rsbrak}[1]{\ensuremath{{}\left.#1\right]}}
\providecommand{\brak}[1]{\ensuremath{\left(#1\right)}}
\providecommand{\lbrak}[1]{\ensuremath{\left(#1\right.}}
\providecommand{\rbrak}[1]{\ensuremath{\left.#1\right)}}
\providecommand{\cbrak}[1]{\ensuremath{\left\{#1\right\}}}
\providecommand{\lcbrak}[1]{\ensuremath{\left\{#1\right.}}
\providecommand{\rcbrak}[1]{\ensuremath{\left.#1\right\}}}
\theoremstyle{remark}
\newtheorem{rem}{Remark}
\newcommand{\sgn}{\mathop{\mathrm{sgn}}}
\providecommand{\abs}[1]{\left\vert#1\right\vert}
\providecommand{\res}[1]{\Res\displaylimits_{#1}} 
\providecommand{\norm}[1]{\left\lVert#1\right\rVert}
%\providecommand{\norm}[1]{\lVert#1\rVert}
\providecommand{\mtx}[1]{\mathbf{#1}}
\providecommand{\mean}[1]{E\left[ #1 \right]}
\providecommand{\fourier}{\overset{\mathcal{F}}{ \rightleftharpoons}}
%\providecommand{\hilbert}{\overset{\mathcal{H}}{ \rightleftharpoons}}
\providecommand{\system}{\overset{\mathcal{H}}{ \longleftrightarrow}}
	%\newcommand{\solution}[2]{\textbf{Solution:}{#1}}
\newcommand{\solution}{\noindent \textbf{Solution: }}
\newcommand{\cosec}{\,\text{cosec}\,}
\providecommand{\dec}[2]{\ensuremath{\overset{#1}{\underset{#2}{\gtrless}}}}
\newcommand{\myvec}[1]{\ensuremath{\begin{pmatrix}#1\end{pmatrix}}}
\newcommand{\mydet}[1]{\ensuremath{\begin{vmatrix}#1\end{vmatrix}}}
\numberwithin{equation}{subsection}
\makeatletter
\@addtoreset{figure}{problem}
\makeatother
\let\StandardTheFigure\thefigure
\let\vec\mathbf
\renewcommand{\thefigure}{\theproblem}
\def\putbox#1#2#3{\makebox[0in][l]{\makebox[#1][l]{}\raisebox{\baselineskip}[0in][0in]{\raisebox{#2}[0in][0in]{#3}}}}
     \def\rightbox#1{\makebox[0in][r]{#1}}
     \def\centbox#1{\makebox[0in]{#1}}
     \def\topbox#1{\raisebox{-\baselineskip}[0in][0in]{#1}}
     \def\midbox#1{\raisebox{-0.5\baselineskip}[0in][0in]{#1}}
\vspace{3cm}
\title{Assignment-18}
\author{Ankur Aditya - EE20RESCH11010}
\maketitle
\newpage
\bigskip
\renewcommand{\thefigure}{\theenumi}
\renewcommand{\thetable}{\theenumi}

\begin{abstract}
This document contains the problem related to characteristic and minimal polynomial (Hoffman Page-198, Q-2) 
\end{abstract}
Download the latex-file from 
\begin{lstlisting}
https://github.com/ankuraditya13/EE5609-Assignment18
\end{lstlisting}

\section{Problem}
Let a, b and c be the elements of a field $\vec{F}$, and let $\vec{A}$ be the following 3$\times$ 3 matrix over $\vec{F}$
\begin{align}
\vec{A} = \myvec{0&0&c\\1&0&b\\0&1&a}
\label{Q}
\end{align}
Prove that the characteristic polynomial for $\vec{A}$ is $x^3-ax^2-bx-c$ and that this is also minimal polynomial for $\vec{A}$.
\section{\textbf{Definitions}}
Minimal polynomial of $\vec{A}$ is a polynomial which satisfies,
\begin{enumerate}
\item[1)] P($\vec{A}$) = 0
\item[2)] P(x) is monic.
\item[3)] It there is some other annihilating polynomial q(x) such that, q($\vec{A}$) = 0, then q does not divide p. 
\end{enumerate}
\section{Solution}
The characteristic polynomial is calculated by solving $\mydet{\vec{A}-\lambda \vec{I}} = 0$
\begin{align}
\implies \mydet{\vec{A}-\lambda \vec{I}} = \mydet{-\lambda &0&c\\1&-\lambda &b\\0&1&a-\lambda}\\
\xleftrightarrow[]{R_2\leftarrow R_2+\lambda R_3} \mydet{-\lambda &0&c\\1&0&b+a\lambda-\lambda^2\\0&1&a-\lambda}\\
\implies \mydet{\vec{A}-\lambda \vec{I}} = 1\mydet{-\lambda & c \\1 & b+a\lambda-\lambda^2}\\
\implies \mydet{\vec{A}-\lambda \vec{I}} = (-\lambda)( b+a\lambda-\lambda^2) - c
\end{align}
Hence the characteristic polynomial of $\vec{A}$ is,
\begin{align}
\lambda^3-a\lambda^2-b\lambda-c
\label{cp}
\end{align}
Now for any r,s $\in \vec{F}$. Also considering the Annihilating polynomial p have degree 2.
\begin{align}
p(\vec{A}) = \vec{A}^2+r\vec{A}+s
\end{align}
\begin{align}
\implies \myvec{0&c&ac\\0&b&c+ba\\1&a&b+a^2} + \myvec{0&0&rc\\r&0&rb\\0&r&ra}+ \myvec{s&0&0\\0&s&0\\0&0&s}
\end{align}
\begin{align}
\therefore f(\vec{A}) = \vec{A}^2+r\vec{A}+s = \myvec{s&c&ac+rc\\r&b+s&c+ba+br\\1&a+r&b+a^2+ra+s} \neq 0
\label{fa}
\end{align}
Hence clearly $f(\vec{A})\neq 0 \forall p \in \vec{F}$, such that degree($\vec{F}$) = 2. Hence for sure minimal polynomial does not have degree 2. Also we know that minimal polynomial can have degree $\leq 3$. Hence degree of minimal polynomial is 3. Also $x^3-ax^2-bx-c$ divides p. Hence from definition-1 we can conclude that,
\begin{align}
p(x) = x^3-ax^2-bx-c
\end{align} 
is a minimal polynomial.
\section{Example}
Let a = 0,b = 0,c = 0 $\in \vec{F}$. Hence,
\begin{align}
\vec{A} = \myvec{0&0&0\\1&0&0\\0&1&0}
\end{align}
Now finding characteristic polynomial by substituting the values of a,b, and c in equation \eqref{cp} we get,
\begin{align}
\lambda^3 = 0
\end{align}
Now let r=0, s=0 $\in \vec{F}$,
Hence f($\vec{A}$) is given by using the equation \eqref{fa},
\begin{align}
\implies f(\vec{A}) = \vec{A}^2+0.\vec{A}+0 = \myvec{0&0&0\\0&0&0\\1&0&0} \neq 0
\end{align}
Hence, $f(\vec{A})\neq 0$, Hence degree of minimal polynomial is 3 and is equal to,
\begin{align}
p(x) = x^3
\end{align}
Verification by calculating p($\vec{A}$),
\begin{align}
p(\vec{A}) = \vec{A}^3 = \myvec{0&0&0\\1&0&0\\0&1&0}\myvec{0&0&0\\1&0&0\\0&1&0}\myvec{0&0&0\\1&0&0\\0&1&0}
\end{align}
\begin{align}
\implies \myvec{0&0&0\\0&0&0\\1&0&0}\myvec{0&0&0\\1&0&0\\0&1&0}\\
\implies f(\vec{A}) = \myvec{0&0&0\\0&0&0\\0&0&0} = 0
\end{align}
Hence, from definition-1 it can be concluded that p(x) = $x^3$ is a minimal polynomial.
\section{Conclusion}
For the $\vec{A}$ given by equation \eqref{Q} ,characteristic and minimal polynomial is given by,
\begin{align}
x^3-ax^2-bx-c
\end{align}
\begin{center}
\textbf{Hence, Proved}
\end{center}
\end{document}